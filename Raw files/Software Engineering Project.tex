\documentclass[14pt]{article}

\usepackage[margin=1in]{geometry}
\usepackage[utf8]{inputenc}
\usepackage{graphicx}
\usepackage{vhistory}

\title{Software Engineering Project\\Hospital Management System}
\author{Tareq Kirresh(TK),Alaa Shuqair(AS), Dua Abu Safiah(DAS), Ashjan Bakri(AB)}
\begin{document}
\begin{figure}[h!]
\centering 
  \includegraphics[width=\textwidth,height=\textheight,keepaspectratio]{Cover.pdf}
\end{figure}
\newpage
\begin{figure}
\centering
  \includegraphics[width=7cm]{LOGO.png}
\end{figure}
\maketitle
\newpage
\tableofcontents 
\newpage 

\begin{versionhistory}
\addcontentsline{toc}{section}{Version History}
  \vhEntry{1.0}{30.07.17}{TK|AS|DAS|AB}{Created.}
  \vhEntry{1.1}{12.08.17}{TK}{Added version history.}
  \vhEntry{1.1}{12.08.17}{AS}{Fixed "use case diagram".}
  \vhEntry{1.1}{12.08.17}{TK}{Added use case specification for "Give Prescription" and "Report to Doctor".}
  \vhEntry{1.1}{12.08.17}{TK|DAS}{Changed order of "Use Cases" subsections.}
  \vhEntry{1.1}{14.08.17}{AS|AB}{Created Sequence Diagram for main flow of use case "Report to Doctor".}
  \vhEntry{1.1}{14.08.17}{DAS|AB}{Created "Concepteual Class Diagram".}
  \vhEntry{1.1}{15.08.17}{TK}{Added Section "Classes".}
  \vhEntry{1.1}{15.08.17}{TK}{Added Image Captions.}
  \vhEntry{1.2}{22.08.17}{DAS}{Added Deployment Diagram.}
  \vhEntry{1.2}{22.08.17}{AS}{Architechture Layer Diagram.}
  \vhEntry{1.2}{23.08.17}{TK|AB}{Added Block Diagram.}
\end{versionhistory}
\newpage
\section{Introduction}
\subsection{State of the Customer}
The Customer is a Hospital, and currently, is completely dependent on a pen-and-paper process to perform all of its management. Our goal
is to transfer this process to an electronic one and streamline it, further adding features that allow the patient to be more 
directly involved in their treatment process. 
\subsection{Current Problems}
As stated above, the customer is still using a paper process for management. Due to the large number of patients and doctors, this 
has become very cumbersome on the customer for the following reasons:
\begin{enumerate}

	
	\item Breakdown in communication: Due to the fact that everything is being done by paper, sometimes the different departments
	do not get their message across clearly due to human errors in spelling or handwriting, leading to the need to repeat work.
	
	\item Theft: Due to the nature of the customer's business, no patient is turned away, and some are lying and claiming prescriptions
	and tests that were not written for them, costing the customer money, and due to the fact that the stock system is manual, the customer
	suspects that some things are taken out without permission. The customer would like to be able to control stocks better.
	
	\item Stocks: Due to the fact that there is no active change in stocks, the customer often finds themselves out of necessary items.
	
	\item Protocol and Chain of command: The customer has said that due to the high amount of personal interaction and the slow speed of the 
	current system, Protocols are being breached often and the Chain of command is not being adhered to in order to provide care in a 
	timely fashion. 
	
	\item Storage issues: Due to the large amounts of records they have, the customer is running out of physical space to store 
	records of their operation. This will lead them to either rent an archive space or to destroy old records.
	
	\item Pollution: Since the customer still uses paper, this paper will eventually have to be destroyed, which is currently 
	through burning. The customer has stated that they would like to be more environmentally-friendly and cut waste.
	
	
\end{enumerate}
\newpage
\section{Requirements}
\subsection{Functional Requirements}
\subsubsection{User Requirements}
\begin{enumerate}

\item The system shall allow patients to sign up an account.

\item The system shall have a log in for all users of all types. For hospital staff, there will also be a specialized page for 
them to log onto work.

	
\item The system shall allow the patient to ask for appointments, communicate with his doctor, see the doctor's notes and
prescriptions , and pay his bills via bank transfer or visa.

\item The system shall allow the receptionist  to register patients, make appointments, and search for information about a patient.

\item The system shall allow the doctor to see patient details and stats,enter a diagnosis for a patient, give them prescriptions, nominate
them for surgery, request tests from the laboratory, reserve the operation theater, and add special notes about the patient.

\item The system shall allow the nurse to see what the doctor's notes are for a patient,
ask the pharmacy for medicine, update patient status, and page for a patient's doctor.

\item The system shall allow the pharmacist to withdraw medicine(based on doctor requests or prescriptions), maintain stocks(count current
stocks and order new stock), view what standing prescriptions and past prescriptions the patient has, and produce stock reports.

\item The system shall allow the laboratory tech to see what tests were requested for a patient, enter the test results,
and maintain laboratory stocks.

\item The system shall connect to the current accounting system at the hospital, and the accounting system will be able to see
all relevant data to produce a bill and allow the patient to pay it as stated above. Patient billing data will also be produced based
on the accounting system

\item The system shall allow the admin to manage all accounts(create/delete/update), modify all data, see all reports,
view all bills, and see all stocks.
\end{enumerate}
\newpage
\subsubsection{System Requirements}
\begin{enumerate}
	\item \begin{enumerate}
			\item Upon opening the appropriate browser page, there will be 2 buttons, one will be sign up, the other will be log in.
			\item Upon opening the sign up page, the patient will be asked to enter their first name, last name, phone number,
			government ID number, email, and a password.
			\item Upon successful sign up, the user will be transfered to the login page.
			\item doctors, pharmacists, lab techs, and nurses are signed up by a special page available only to the admin and are assigned
			an account type.
			\end{enumerate}
	\item \begin{enumerate}
			\item Upon opening the appropriate browser page, there will be 2 buttons, one will be sign up, the other will be log in.
			\item Upon opening the log in page, the user will be asked to enter their email and passwords. Logins detected from 
			inside the hospital LAN that are from the @hospital.com domain will be redirected to a page appropriate to the account type
			assigned to them (eg, doctors are taken to their homepage, nurses to another, etc). Patients will be taken to a dash. Staff will 
			be taken to a page for them to log onto work. after they press log onto work,
			the time is sent to the HR system(that they currently use),then they are redirected to their main page.
		  \end{enumerate}
		  
	\item \begin{enumerate}
			\item The patient dash is a screen where the patient is given the most relevant information about them.
			\item The patient is shown any upcoming appointments, any outstanding prescriptions or bills, and any doctor orders.
			\item There will be a button that allows the patient to create a new appointment with a doctor
			\item There will be a button that allows the patient to start a chat with their doctor or call them via VOIP if they are 
			available, 
			\item There will be a payment screen for any outstanding bills, which is done through a visa or a bank account transfer. 
			\item Additionally, there will be screens for a patient to view their billing history, diagnosis history, prescription history,
			and appointment history.
		  \end{enumerate}
		  
	\item \begin{enumerate}
			\item the receptionist main page has the buttons and menus for the receptionist to 
				\begin{enumerate}
					\item Create a new user for a patient.
					\item Search for an already registered patient.
					\item Look at a patient's data.
					\item Create an appointment for a patient. 
					\item message and call other hospital staff.
				\end{enumerate}
			\item sign out of work and log off.
		  \end{enumerate}
		  
		\item \begin{enumerate}
			\item the doctor main page has the buttons and menus for the doctor to 
				\begin{enumerate}
					\item See their schedule.
					\item Look at a patient's data.
					\item Place their notes about the patient.
					\item Create a new prescription for a patient.
					\item Request a test for a patient.
					\item Nominate a patient for surgery and reserve the operating theater. 
					\item message and call other hospital staff.
				\end{enumerate}
			\item sign out of work and log off.
		  \end{enumerate}
		\item \begin{enumerate}
			\item the nurse main page has the buttons and menus for the nurse to 
				\begin{enumerate}
					\item Look at a patient's data.
					\item Read the doctor's notes about a patient.
					\item Request Medicine for a patient.
					\item Update the patient's status, in that, the time they were checked in, what ward they are in, what time
					they were given medicine, if they are deteriorating or doing fine, and if they have given them something other
					than what the doctor has ordered.
					\item Emergency page the doctor assigned to the patient(on their pager, not on the system).
					\item message and call other hospital staff.
				\end{enumerate}
			\item sign out of work and log off.
		  \end{enumerate}
	\item \begin{enumerate}
		\item the pharmacist main page has the buttons and menus for the pharmacist to 
				\begin{enumerate}
					\item Look at a patient's data.
					\item Read the doctor's notes about a patient.
					\item withdraw medicine for a patient from the stock based on their 
					prescription . The amount available in stock will go down accordingly. if the
					stock is at a critical level(less than 15 \% of normal stock), then the pharmacist will be notified.
					\item Place orders for new stock, and send them to purchasing department .
					\item Produce stock reports, which are a list of every medicine available by its brand name, current stock, and date. 
					Each entry in the list has a sublist of the time and amount of every withdrawal, along with which patient it was withdrawn
					for and which pharmacist withdrew it.
					\item message and call other hospital staff.
				\end{enumerate}
			\item sign out of work and log off.
		  \end{enumerate}
		  
		\item \begin{enumerate}
		\item the lab tech main page has the buttons and menus for the lab tech to 
				\begin{enumerate}
					\item Look at a patient's data.
					\item Read the doctor's notes about a patient.
					\item See what tests the doctor has ordered for a patient.
					\item Withdraw items used in tests(eg test tubes, litmus paper, etc), and be notified whenever the
					number is below a threshold(less than 15\% of the normal stock).
					\item Enter the results of the tests he has performed. Each test will have its own page depending on the test. 
					we are currently working together with the lab techs to create the form needed to be filled out for each test.
					\item Produce stock reports, which are a list of every perishable 
					item used in tests available by its brand name, current stock, and date. 
					Each entry in the list has a sublist of the time and amount of every withdrawal, along with which patient it
					was withdrawn for and which technician withdrew it.
					\item Order stock and send it to the purchasing department.
					\item message and call other hospital staff.
				\end{enumerate}
			\item sign out of work and log off.
		  \end{enumerate}
		  
		 \item \begin{enumerate}
		\item The accounting system shall be connected to this system by means of the available API. 
		\item Patient billing data in this system is based on the data in the accounting system.
		\item This system allows a patient to pay via visa or bank transfer. 
		\item Note: Cash and cheque payments go through the present accounting system and not this one. 
		  \end{enumerate}
		  
		\item \begin{enumerate}
		\item The admin page has a button that takes the admin to the database management interface(eg PHPmyadmin).
		all modification of data will be done manually from there.
		\item The admin page has buttons that lead to a page where new staff are signed up and assigned an account type.
		\item The admin page has menus that allow the admin to see all reports in the hospital. 
		\item message and call other hospital staff.
		\item sign out of work and log off.
		  \end{enumerate}
		  
		
\end{enumerate}

\subsection{Non-Functional Requirements}
\begin{enumerate}
	\item The system shall be secured, With all passwords at least hashed and traffic being encrypted with at least https.
	\item The system shall guarantee patient-doctor confidentiality by ensuring that only authorized
	persons see specific data related to their work only.
	\item The system must not take more than 1 day of training to learn to use.
	\item The system must be responsive; No request shall take more than 2 seconds to be processed.
	\item The system must have an uptime of at least 99.9 \%.
	\item The system must implement the API's for visa and bank payments and the API's for the accounting and HR softwares.
	\item The system must have integrated VOIP and Pager capability. 
\end{enumerate}
\newpage
\section{Use Cases}
\subsection{Use Case Diagram }
\begin{figure}[h!]
\centering{}
  \caption{Use Case Diagram}
  \includegraphics[width=14.3cm]{hospitalDiagram1fixed.png}
\end{figure}
\newpage
\begin{figure}[h!]
\centering 
  \caption{Use Case Diagram(conts.)}
  \includegraphics[width=14.3cm]{hospitalDiagram-2.pdf}
\end{figure}
\newpage 
\subsection{Use Case Specification}
\subsubsection{Give Prescription}
\begin{itemize}
		\item Primary Actors : 
			\begin{enumerate}
				\item Doctor
				\item Patient
			\end{enumerate}
		\item Preconditions :
			\begin{enumerate}
				\item Patient has an account on the system.
				\item Doctor is logged on and has a selected a patient file.
			\end{enumerate}
		\item Basic Flow of Events :
			\begin{enumerate}
				\item The Doctor selects the "Add Prescription" option.
				\item The System shows a new prescription form.
				\item The Doctor selects the drug, the amount and enters his special notes.
				\item The Doctor selects "Add".
				\item The Systems checks if the patient has any allergies to the selected medicine.
				\item The Systems checks if the patient has any outstanding prescriptions for the selected medicine.
				\item The System adds a new prescription for this patient.
				\item The System Prints a copy of the prescription for the patient. 
			\end{enumerate}
		\item Alternative Flows :
			\begin{enumerate}
				\item The Doctor selects "Cancel" : 
					\begin{enumerate}
						\item The system goes back to the patient file.
					\end{enumerate}
				\item The Patient has an allergy to the selected medicine :
					\begin{enumerate}
						\item A Message is displayed saying "Patient has allergy, Please check selected medicine"
						\item The System returns to the "Add Prescription" form.
					\end{enumerate}
				\item The Patient has an outstanding prescription for the selected medicine :
					\begin{enumerate}
						\item A Message is displayed saying "Patient has outstanding prescriptions, Please check selected medicine" . 
						\item The System returns to the "Add Prescription" form.
					\end{enumerate}
			\end{enumerate}
\end{itemize}
\newpage
\subsubsection{Report to Doctor}
\begin{itemize}
		\item Primary Actors : 
			\begin{enumerate}
				\item Nurse
				\item Doctor
			\end{enumerate}
		\item Preconditions :
			\begin{enumerate}
				\item Patient is assigned to the nurse and the doctor.
				\item Nurse has logged in and selected the Patient's file.
			\end{enumerate}
		\item Basic Flow of Events :
			\begin{enumerate}
				\item Nurse selects the "Create Patient Report" option.
				\item The System shows a new report form.
				\item The Nurse enters the patient vitals, status, and her notes.
				\item The Nurse selects the "Send Report" option.
				\item The System adds the time and date to the report.
				\item The System adds the report to the patient file.
				\item The System notifies the doctor that he has an unseen report.	
			\end{enumerate}
		\item Alternative Flows :
			\begin{enumerate}
				\item The Nurse selects "Cancel" : 
					\begin{enumerate}
						\item The system goes back to the patient file.
					\end{enumerate}
			\end{enumerate}
\end{itemize}
\newpage
\begin{figure}[h!]
\centering
  \caption{Sequence Diagram for Report to Doctor}
  \includegraphics[width=14.5cm]{sequenceDiagram.pdf}
\end{figure}
\newpage 
\section{Classes}
Below is the concepteual class diagram for the system.
\footnote{We are not sure we need this section, but it felt wrong having the conceptual class diagram without a section related
to it}
	\begin{figure}[h!]
		\centering 
		\caption{Conceptual Class Diagram}
		\includegraphics[height=16cm,keepaspectratio]{ConceptualClassDiagram.pdf}
	\end{figure}
\newpage 
\section{Architechture}
	\subsection{Deployment Diagram}
		\begin{figure}[h!]
			\centering 
			\caption{Deployment Diagram}
			\includegraphics[width=15cm,keepaspectratio]{DeploymentDiagram.pdf}
		\end{figure}
\newpage 
	\subsection{Architechture Layer Diagram}
		\begin{figure}[h!]
			\centering 
			\caption{Architechture Layer Diagram}
			\includegraphics[height=16cm,keepaspectratio]{LayeredArchitechtureDiagram.pdf}
		\end{figure}
\newpage
	\subsection{System Block Diagram}
		\begin{figure}[h!]
			\centering 
			\caption{System Block Diagram}
			\includegraphics[width=\textwidth,keepaspectratio]{SystemBlockDiagram.png}
	\end{figure}
\end{document}
